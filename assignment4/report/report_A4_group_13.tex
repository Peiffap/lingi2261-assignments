\documentclass[journal,onecolumn]{IEEEtran}
%\usepackage[left=2.2cm,right=2.2cm,top=2.5cm,bottom=2.5cm]{geometry}
\usepackage[utf8]{inputenc}
\usepackage[english]{babel}
\usepackage{minted}
\usepackage{booktabs}
\usepackage{commath}
\usepackage{float}
\usepackage{mathtools}
\usepackage{color}
\usepackage{amsthm}
\usepackage{parskip}


\usepackage[binary-units=true]{siunitx}

\newcommand{\py}[1]{\mintinline{python}{#1}}

\title{Artificial Intelligence (\texttt{LINGI2261}) \\ Assignment 4 --- Group 13}
\author{Martin Braquet, Gilles Peiffer}
\date{December 11, 2019}

\begin{document}

\maketitle

\section{The Bin Packing Problem}
\begin{enumerate}
	\item The bin packing problem can be formulated as a local search problem as follows: % TODO
	\item The initial solution is constructed as % TODO explain
	
	
	The successor functions works as % TODO explain
	\item % TODO
	\item \begin{enumerate}
		\item % TODO
		\item % TODO
		\item % TODO
		\item % TODO
	\end{enumerate}
\end{enumerate}

\section{Propositional Logic}
\subsection{Models and Logical Connectives}
\begin{enumerate}
	\item The first sentence has ten valid interpretations.
	
	
	The second sentence has four valid interpretations.
	
	
	The third sentence has one valid interpretation.
\end{enumerate}

\subsection{Color Grid Problem}
\begin{enumerate}
	\item % TODO
	\item % TODO
	\item % TODO
\end{enumerate}

\end{document}
